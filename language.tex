\section{Language}
\label{sec:language}

The Modal Language \system{K}{n}{} is equivalent to its set of \emph{well-formed
formulae}, denoted \wff, which is constructed from an enumerable set of
\emph{propositional symbols}:
\begin{equation}
    \label{eq:prop_symb}
    \mathcal{P} = \{p, q, r, \ldots\},
\end{equation}
the negation symbol $\neg$, the disjunction symbol $\lor$ and the modal
connective $\nec{a}$, that expresses the notion of necessity, for each index (or
agent) $a$ in a finite, fixed set:
\begin{equation}
    \label{eq:agents}
    \mathcal{A} = \{1, \ldots, n\}, n \in
    \mathbb{N}
\end{equation}

The propositional symbols combined with the logic operators, represented by the
symbols mentioned above, are arranged to form sentences (also, parentheses can
be used to avoid ambiguity). Therefore, the set of \wff~is recursively defined
as showed in~\ref{def:wff}.

\begin{definition}
\label{def:wff}
    The set of well-formed formulae, \wff, is the least set such that:
    \begin{enumerate}
        \item $\mathcal{P} \subset \wff$
            \vspace{.2ex}
        \item if $\varphi, \psi \in \wff$, then so are $\neg \varphi, (\varphi
            \lor \psi)$ and $\nec{a} \varphi$, for each $a \in \Agents$
    \end{enumerate}
\end{definition}

As one might know, other logic operators may be introduced as abbreviation to
formulae constructed using the operators defined as the usual. In particular,
this paper considers the following abbreviations:
\begin{enumerate}
    \item $\varphi \wedge \psi = \neg(\neg \varphi \lor \neg \psi)$
        (conjuction) 
    \item $\varphi \then \psi = \neg \varphi \lor \psi$ (implication)
    \item $\varphi \iff \psi = (\varphi \then \psi) \land (\psi \then \varphi)$
        (equivalence)
    \item $\pos{a} \varphi = \neg \nec{a} \neg \varphi$ (possibility)
    \item $\textbf{false} = \varphi \wedge \neg \varphi$ (\emph{falsum})
    \item $ \textbf{true} = \neg \textbf{false}$ (\emph{verum})
\end{enumerate}
And the precedence of the operators is as follows: 
\begin{enumerate}
    \item $\neg, \nec{a}, \pos{a}$
    \item $\wedge$
    \item $\lor$ 
    \item $\then, \iff$
\end{enumerate}

Logics that involve $n$ agents in the modal logic, with $n \in \Nat$, are know
as Multimodal Logics. When $n = 1$, we often omit the index in the modal
operators, i.e., we just write $\nec{}$ and $\pos{}$.


The maximal number of modal operators in a formula is defined as its \emph{modal
depth} and denoted $mdepth$. The maximal number of modal operators in which
scope the formula occurs is defined as the \emph{modal level} of that formula,
and it is denoted $mlevel$.
For instance, in $\nec{a}\pos{a} p$, $mdepth(p) = 0$ and $mlevel(p) = 2$.

\subsection{Semantics}
\label{sec:semantics}

\begin{definition}
    A Kripke model for the set of propositional symbols \prop~and the agents
    \Agents~is given by the tuple $\Model = (W, \st_0, R_1, \ldots, R_n, \pi)$,
    where $W$ is a non-empty set of possible worlds with a distinguinshed world
    $\st_0$, each $R_a, a \in \Agents$, is a binary relation on $W$, and $\pi: W
    \times \prop \longrightarrow \{false, true\}$ is the valuation function that
    associates to each world $w \in W$ a truth-assignment to propositional
    symbols.
\end{definition}

From the definition of a Kripke model, one can define the satisfiability and
validity of a formula in \system{K}{n}{}.

\begin{definition}
    Let $\Model = (W, \st_0, R_1, \ldots, R_n, \pi)$ be a Kripke model for
    \prop~and \Agents, and consider $\st \in W, p \in \prop$ and $\formula,
    \psi \in \wff$. The \emph{satisfiability relation}, denoted by 
    \sat{\Model}{\st}{\formula}, among the world \st~and a formula \formula~in
    the model \Model,
    is inductively defined by:
    \begin{enumerate}
        \item \sat{\Model}{\st}{p} if, and only if, $\pi(\st, p) = \textbf{true}$ 
        \item \sat{\Model}{\st}{\neg\formula} if, and only if, $
            \langle \Model, \st \rangle \not \models \formula$
        \item \sat{\Model}{\st}{\formula\lor\psi} if, and only if,
            \sat{\Model}{\st}{\formula} or \sat{\Model}{\st}{\psi}
        \item \sat{\Model}{\st}{\nec{a} \formula} if, and only if, $\forall~t\in
            W$, with $a \in \Agents$, $(\st, t) \in R_a$ implies
            \sat{\Model}{t}{\formula}
    \end{enumerate}
\end{definition}

A set of formulae $\Gamma = \{\gamma_1, \ldots, \gamma_r\}, r \in \Nat$, is
satisfiable in a world \st~if, and only if, each of its formulae is satisfiable
in this world, i.e., \sat{\Model}{\st}{\Gamma}
$\iff$ \sat{\Model}{\st}{\gamma_1\land\ldots\land\gamma_r}

\begin{definition} % SATISFIABILITY
    A formula $\formula \in \wff$ is said to be \emph{satisfiable} if exists a
    Kripke model \Model~such that \sat{\Model}{\st_0}{\formula}.
\end{definition}

\begin{definition} % VALIDITY
    A formula $\formula \in \wff$ is said to be \emph{valid} if for all Kripke
    model \Model, we have that \sat{\Model}{\st_0}{\formula}.
\end{definition}

When one is considering a set of formulae instead of a single one, both
definitions of satisfiability and validity holds similarly to the definition of
satisfiability relation.
